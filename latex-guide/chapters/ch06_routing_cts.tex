\chapter{Routing and Clock Tree Synthesis (CTS)}

\section{Overview}
In this final implementation chapter, we complete the RTL-to-GDSII flow. Following the placement of standard cells, we must now:
\begin{enumerate}
    \item \textbf{Clock Tree Synthesis (CTS):} Distribute the clock signal to all sequential elements (Flip-Flops) with minimum skew and latency.
    \item \textbf{Routing:} Connect the signal pins using metal wires. This is performed in two stages:
    \begin{itemize}
        \item \textbf{Global Routing:} Allocates routing resources and plans paths to manage congestion.
        \item \textbf{Detailed Routing:} Actually lays down the wires and vias, checking for Design Rule (DRC) violations.
    \end{itemize}
    \item \textbf{Final DEF Export:} Generating the output file with the fully routed design.
\end{enumerate}

\section{Files Required}
The \texttt{RTL2GDS\_WS\_Files/openroad/} directory contains \texttt{run\_routing.tcl} which performs all the steps below:
\begin{itemize}
    \item Loads the placed design from \texttt{design6\_placed.def}
    \item Performs Clock Tree Synthesis (CTS)
    \item Runs global and detailed routing
    \item Generates timing reports and final DEF output
\end{itemize}

\section{Steps}

\subsection{1. Execute the Routing Flow}
Navigate to \texttt{RTL2GDS\_WS\_Files/openroad/} and execute the script:
\begin{lstlisting}[language=bash]
cd RTL2GDS_WS_Files/openroad
openroad -exit run_routing.tcl
\end{lstlisting}

For visualization during the flow, use the GUI mode:
\begin{lstlisting}[language=bash]
openroad -gui
% source run_routing.tcl
\end{lstlisting}

\subsection{2. Analyzing CTS Results}
Once the script reaches the CTS stage, the tool reports:
\begin{itemize}
    \item Number of clock buffers inserted (typically 2-4 for small designs)
    \item Clock tree depth (number of buffer levels)
    \item Average sink wire length
    \item Path depth distribution
\end{itemize}

In GUI mode, use the \textbf{Clock Tree Viewer} from the "Windows" menu to visualize the tree structure showing the clock root driving buffers which then drive the Flip-Flops.

\subsection{3. Analyzing Routing Results}
After detailed routing completes:
\begin{itemize}
    \item \textbf{Signal Nets:} Enable "Nets" in the Display Control panel to see the metal traces connecting the cells.
    \item \textbf{Heat Maps:} Use the "Heat Maps" tool to inspect \textbf{Routing Congestion}. Red areas indicate high congestion where the router struggled to find paths.
    \item \textbf{DRC Report:} Check \texttt{design6.drc} for any design rule violations.
\end{itemize}

\section{Final Product: Layout View}
The final output of this flow is the \textbf{DEF} file (\texttt{design6\_final.def}) which contains the complete routed design.

To visualize the final layout:
\begin{enumerate}
    \item In OpenROAD GUI, ensure all layers (met1-met5, vias) are visible.
    \item Zoom out to see the entire die boundary, including the IO pins and the Power Distribution Network.
    \item You can see the placed cells, power grid, and routing on multiple metal layers.
\end{enumerate}

\begin{figure}[htbp]
    \centering
    \includegraphics[width=0.85\textwidth]{images/ch06_routed.png}
    \caption{Final routed design in OpenROAD GUI showing the complete physical implementation. The layout displays all metal layers (met1-met5) with signal routing, clock tree buffers, power distribution network, and fully connected standard cells. Different colors represent different metal layers and routing resources.}
    \label{fig:ch06_routed}
\end{figure}

\textbf{Note:} To convert the DEF file to GDSII format for fabrication, you would typically use KLayout or Magic with the appropriate technology files.

\section{Deliverables}
\begin{itemize}
    \item \textbf{Routed Design:} \texttt{design6\_final.def}
    \item \textbf{DRC Report:} \texttt{design6.drc}
    \item \textbf{Routing Guide:} \texttt{design6.guide}
    \item \textbf{Timing Reports:} Final setup/hold timing with routed parasitics
\end{itemize}