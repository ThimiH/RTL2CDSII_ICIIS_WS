\chapter{Routing and Clock Tree Synthesis (CTS)}

\section{Overview}
In this final implementation chapter, we complete the RTL-to-GDS flow. Following the placement of standard cells, we must now:
\begin{enumerate}
    \item \textbf{Clock Tree Synthesis (CTS):} Distribute the clock signal to all sequential elements (Flip-Flops) with minimum skew and latency.
    \item \textbf{Routing:} Connect the signal pins using metal wires. This is performed in two stages:
    \begin{itemize}
        \item \textbf{Global Routing:} Allocates routing resources and plans paths to manage congestion.
        \item \textbf{Detailed Routing:} Actually lays down the wires and vias, checking for Design Rule (DRC) violations.
    \end{itemize}
    \item \textbf{Final GDSII Export:} Generating the binary file for manufacturing.
\end{enumerate}

\section{Files to Add}
The \texttt{openroad/} directory already provides \texttt{run\_routing.tcl} alongside the placement scripts, so you can reuse it directly.

\section{Steps}

\subsection{1. Execute the Routing Flow}
Navigate to \texttt{RTL2GDS\_WS\_Files/openroad/} and execute the script.
\begin{lstlisting}[language=bash]
openroad -gui
% source run_routing.tcl
\end{lstlisting}

\subsection{2. Analyzing CTS Results}
Once the script reaches the CTS stage, use the \textbf{Clock Tree Viewer} in the GUI:
\begin{itemize}
    \item Go to the "Windows" menu or the top-right pane to find the Clock Tree Viewer.
    \item Click "Update" to visualize the tree structure. You will see the clock root driving buffers (inverters), which then drive the Flip-Flops.
\end{itemize}

\subsection{3. Analyzing Routing Results}
After detailed routing completes:
\begin{itemize}
    \item \textbf{Signal Nets:} Enable "Nets" in the Display Control panel to see the metal traces connecting the cells.
    \item \textbf{Heat Maps:} Use the "Heat Maps" tool to inspect \textbf{Routing Congestion}. Red areas indicate high congestion where the router struggled to find paths.
\end{itemize}

\section{Final Product: IO Package View}
The final output of this flow is the \textbf{GDSII} file (\texttt{design6\_final.gds}).

To visualize the final "Chip" view:
\begin{enumerate}
    \item In OpenROAD, ensure all layers (Metal1-Metal10, Vias) are visible.
    \item Zoom out to see the entire die boundary, including the IO pins and the Power Ring.
    \item Take a screenshot of this view.
\end{enumerate}

\textit{Note: Open-source tools typically generate the Die Layout. For a full 3D IO-Package wire-bond visualization (as requested in the introduction), please refer to the Appendix for instructions using Synopsys tools, as OpenROAD focuses on the die-level implementation.}

\section{Deliverables}
\begin{itemize}
    \item \textbf{Final Layout Screenshot:} Showing fully routed standard cells and power grid.
    \item \textbf{Clock Tree Plot:} Screenshot from the Clock Tree Viewer.
    \item \textbf{Final GDS File:} \texttt{design6\_final.gds}.
\end{itemize}