\chapter{Static Timing Analysis (STA)}

\section{Overview}
In this chapter, we verify that our synthesized netlist meets timing requirements using **OpenSTA**. We will check for:
\begin{itemize}
    \item \textbf{Setup Violations:} Can the signal propagate through the logic within one clock cycle?
    \item \textbf{Hold Violations:} Does the signal stay stable long enough after the clock edge?
\end{itemize}

\section{Installation}
We will install OpenSTA by building it from source.
\subsection{1. Install Dependencies}
Ensure you have the build tools installed (CMake, Make, GCC/G++). If not already installed:
\begin{lstlisting}[language=bash]
sudo apt-get install cmake clang bison flex zlib1g-dev
\end{lstlisting}

\subsection{2. Build OpenSTA}
Clone the repository and build using CMake.
\begin{lstlisting}[language=bash]
# 1. Clone the repository
git clone https://github.com/The-OpenROAD-Project/OpenSTA.git
cd OpenSTA

# 2. Create build directory
mkdir build
cd build

# 3. Configure and Compile
cmake ..
make

# 4. Install
sudo make install
\end{lstlisting}

\subsection{3. Verify Installation}
Type \texttt{sta} in your terminal. You should see the OpenSTA prompt. Type \texttt{exit} to quit.

\section{Files to Add}
We need to create two new files: a Synopsys Design Constraint (SDC) file to define our timing requirements, and a TCL script to run the analysis.

\subsection{1. Constraint File (sta/design6.sdc)}
The bundle already includes \texttt{RTL2GDS\_WS\_Files/sta/design6.sdc}, which defines a 10ns (100 MHz) clock period.

\begin{lstlisting}[language=tcl, caption=Timing Constraints (design6.sdc)]
# 1. Define the Clock
# Period: 10ns (100 MHz), applied to port 'clk'
create_clock -name clk -period 10 [get_ports clk]

# 2. Input Delays
# Assume inputs arrive 2ns after the previous clock edge
set_input_delay -clock clk 2 [get_ports {A B C D start rst_n}]

# 3. Output Delays
# Assume outputs must be stable 2ns before the next clock edge
set_output_delay -clock clk 2 [get_ports {F valid}]
\end{lstlisting}

\subsection{2. STA Script (\texorpdfstring{\texttt{sta/run\_sta.tcl}}{sta/run_sta.tcl})}
In the same directory, \texttt{run\_sta.tcl} automates the OpenSTA session by loading the SkyWater library, synthesized netlist, and constraints.

\begin{lstlisting}[language=tcl, caption=STA Run Script (run\_sta.tcl)]
# 1. Read the Standard Cell Library
# (Update path to match your system)
read_liberty ../lib/sky130_fd_sc_hd__tt_025C_1v80.lib

# 2. Read the Synthesized Netlist (from Chapter 3)
read_verilog ../synth/design6_netlist.v

# 3. Link the Design
link_design design6_seq

# 4. Read Constraints
read_sdc design6.sdc

# 5. Report Checks
# Check for Setup Violations (Max Delay)
report_checks -path_delay max -format full

# Check for Hold Violations (Min Delay)
report_checks -path_delay min -format full

# 6. Exit
exit
\end{lstlisting}

\noindent Ensure the path \texttt{../lib/sky130\_fd\_sc\_hd\_\_tt\_025C\_1v80.lib} points to the Liberty file located in \texttt{RTL2GDS\_WS\_Files/lib/}.

\section{Steps}
\begin{enumerate}
    \item Navigate to \texttt{RTL2GDS\_WS\_Files/sta/}.
    \item Run OpenSTA with the script:
\begin{lstlisting}[language=bash]
cd RTL2GDS_WS_Files/sta
sta run_sta.tcl
\end{lstlisting}
\end{enumerate}

\section{Expected Outputs}
You will see detailed timing reports in the terminal showing the critical paths.

\subsection{1. Setup Check (Max Delay)}
Look for the \textbf{Slack (VIOLATED)} or \textbf{Slack (MET)} line in the max delay report.
\begin{itemize}
    \item \textbf{Data Required Time:} Clock Period - Library Setup Time.
    \item \textbf{Data Arrival Time:} Path Delay + Input Delay.
    \item \textbf{Slack:} (Data Required) - (Data Arrival).
\end{itemize}

If Slack is positive, the design meets the setup timing requirement.

\subsection{2. Hold Check (Min Delay)}
Look for the hold check report (min delay).
\begin{itemize}
    \item \textbf{Slack:} (Data Arrival) - (Data Required).
\end{itemize}

If Slack is positive, the design meets hold timing requirement.