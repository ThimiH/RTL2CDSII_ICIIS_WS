\chapter{Placement and Power Planning}

\section{Overview}
In this chapter, we move from the logical world (netlists) to the physical world. We will use **OpenROAD** to perform:
\begin{itemize}
    \item \textbf{Floorplanning:} Defining the die area and core size.
    \item \textbf{Power Planning (PDN):} Creating the power ring and grid (VDD/VSS).
    \item \textbf{Placement:} Placing the standard cells into rows, first roughly (Global) and then precisely (Detailed).
\end{itemize}

\section{Installation}
We will install the OpenROAD application locally, following the steps from the NPTEL tutorial.

\subsection{1. Clone Repository}
OpenROAD requires a recursive clone to fetch all submodules.
\begin{lstlisting}[language=bash]
git clone --recursive https://github.com/The-OpenROAD-Project/OpenROAD.git
cd OpenROAD
\end{lstlisting}

\subsection{2. Install Dependencies}
OpenROAD provides a script to install all required Linux libraries.
\begin{lstlisting}[language=bash]
sudo ./etc/DependencyInstaller.sh
\end{lstlisting}

\subsection{3. Build and Install}
Create a build directory and compile using CMake.
\begin{lstlisting}[language=bash]
mkdir build
cd build
cmake ..
make
sudo make install
\end{lstlisting}
    extit{Note: The compilation process may take considerable time depending on your machine specifications.}

\section{Files to Add}
For physical design, we need specific technology files (LEF) which define the physical dimensions of the standard cells.

\subsection{1. Technology Files}
Ensure the following SkyWater SKY130 HD files are available in \texttt{RTL2GDS\_WS\_Files/lib/} (they are included in the shared workspace):
\begin{itemize}
    \item \texttt{sky130\_v5\_5.lef} (technology + routing layers)
    \item \texttt{output.lef} (standard cell LEF views)
    \item \texttt{sky130\_fd\_sc\_hd\_\_tt\_025C\_1v80.lib} (Liberty timing/power)
\end{itemize}

\subsection{2. Configuration Files}
The \texttt{openroad/} directory in the bundle already includes the automation scripts:
\begin{itemize}
    \item \texttt{openroad/pdn\_config.tcl}: Defines the power grid structure.
    \item \texttt{openroad/run\_placement.tcl}: Drives the OpenROAD placement flow.
\end{itemize}

\section{Steps}

\subsection{1. Run the Placement Flow}
Navigate to the \texttt{openroad/} directory and execute the script:
\begin{lstlisting}[language=bash]
openroad -gui
% source run_placement.tcl
\end{lstlisting}
Using the \texttt{-gui} flag allows you to visualize the changes in real-time.

\subsection{2. Visualization Stages}
As the script runs, you will observe the following stages in the GUI:

\begin{enumerate}
    \item \textbf{Floorplan initialization:} The core boundary is defined.
    \item \textbf{Tap Cell Insertion:} Well-tap cells are added to the boundary to prevent latch-up.
    \item \textbf{PDN Generation:} A power grid (metal tracks for VDD/VSS) is drawn over the core.
    \item \textbf{Global Placement:} Cells are spread out roughly (shown as red blocks), often with some overlaps initially.
    \item \textbf{Detailed Placement:} Cells are snapped into valid rows and overlaps are removed.
\end{enumerate}

\section{Deliverables}
\begin{itemize}
    \item \textbf{Screenshots:} Capture the final layout showing standard cells placed inside the core area.
    \item \textbf{DEF File:} The script generates \texttt{design6\_placed.def}, which contains the physical location of every cell.
\end{itemize}